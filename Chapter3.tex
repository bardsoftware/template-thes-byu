\chapter{Clear Thinking}
\label{chap:ClearThinking}

\section{Creating an outline}
\label{sec:Outline} \index{Outline}

Once you identify your research project, you should immediately
begin the writing process. The initial writing process may begin in
your lab notebook. (A good physicist keeps a lab notebook or
journal.) Make an outline (as best you can envision) of your thesis.
An example of such an outline is shown in Table~\ref{table:outline}.
In this example, the primary student research project is represented
in section~2.3. The student will work on a spectrometer which is
just one part of a larger research project involving other group
members. Nevertheless, the thesis will encompass the overall purpose
and results of the entire project (even though this will overlap
with theses written by other students). Other students, for example,
may have developed the equipment described in sections~2.1 or 2.2,
and so these items would naturally be emphasized in detail by them
(in their theses).

\begin{table}
 \renewcommand{\baselinestretch}{1}\scriptsize
\textbf{Title:} Development of a Spectrometer to Study the Influence
of Counter-Propagating Light on High-Order Harmonic Generation

\scriptsize
\begin{tabular}{p{1.8in}p{1.8in}p{1.8in}}

  %\hline
  \\
  \textbf{Chapter 1:} Introduction  & \textbf{Chapter 2:} Experimental Setup & \textbf{Chapter 3:} Results \\
  & & \\
  \raggedright

  1.1 Overview

  1.2 Background

  1.2.1 Laser Harmonic Generation

  1.2.2 High Harmonic Generation

  1.2.3 Phase Matching and Conversion Efficiency

  1.3 Using Counter-Propagating Light to Manipulate Phase Matching

  &

  \raggedright

  2.1 Laser System and Pulse Characteristics

  2.2 Experimental Setup

  2.3 High Harmonics Spectrometer

  2.3.1 Design Overview

  2.3.2 Diffraction Grating

  2.3.3 Imaging Issues

  2.3.4 Positioning Controls

  2.4 Detection

  &

  \begin{raggedright}

  3.1 Measurement of High Harmonics

  3.2 Influence of Counter-Propagating Light

  3.3 Interpretation of Results

  3.4 Conclusion and Future Outlook

  \end{raggedright}

  \\ & & \\
\end{tabular}
\renewcommand{\baselinestretch}{1.66}\small\normalsize
\caption{\label{table:outline} Sample outline for a senior thesis.}
\end{table}

The overview section (1.1) should motivate the reason for the
research without relying on specific background that will be
introduced in the later sections. You should include general
motivational statements (that you might give, for example, to a
science news reporter) as in the following example: ``Laser
high-order harmonic generation is a unique source of directional and
bright extreme ultraviolet radiation (EUV). This short wavelength
light source may have future applications in ultrafine resolution
photolithography. Because the high harmonics are coherent and
generated with short pulse lasers, they can be used to probe
ultrafast phenomena when high-energy photons are needed..." The
overview should also provide the reader with a clear demarcation of
the scope of your contribution to the overall project.

Near the end of Chapter~1, as you narrow to the specific problem to
be addressed, be sure to provide a more detailed outline of the
overall project than was given in the opening section. Again, be
specific about your role in the project. Briefly introduce and
summarize what will be discussed in the remainder of the thesis. As
is evident, the content of the first chapter depends strongly on
what is written in subsequent chapters. Therefore, the first chapter
is typically the last chapter to be completed. Nevertheless, it
should also be one of the first chapters that you begin to write.

It may happen that, while a student makes a meaningful contribution
to the overall project, the final results are not obtained before
the senior thesis is submitted. This is not ideal, but this happens
to students quite often. In this situation, the final chapter might
be entitled ``Discussion." If attempts were made at obtaining data,
the concluding chapter might describe the reasons (especially
reasons involving physics) why the attempts were unsuccessful. The
final chapter should provide suggestions on how to remedy the
situation (which can be helpful to future students who continue the
project). If the data-taking stage is not reached, the concluding
chapter might describe preliminary checks of equipment and expected
roles for the new equipment in the overall project.

The format suggested in Table~\ref{table:outline} is very flexible
and would likely be somewhat different, for example, in the case of
a thesis based on theoretical work. You should decide what works
best for you in consultation with your advisor. You may want to
examine several senior theses written by previous students which are
available in the physics department reading room (N288). There are
examples of good and not-so-good theses there. Exemplary theses were
written by Shannon Lunt, Deborah Paulsen, and Michael Ware.

\section{Importance of feedback}
\label{sec:Feedback}

Obtain feedback at every step of the writing process. Go over your
outline with your advisor. It is much less painful to rearrange or
to delete sections before you write them, when they are represented
merely in outline form. Make brief notes indicating what will go
into each section (e.g. a summary of research by Group X and Y, a
schematic of an experimental setup, a blowup view of a critical
part, etc.). Discuss your initial ideas and brainstorm together with
your advisor.

You should start writing portions of your thesis early, even though
some sections will need to wait until after the research is
concluded. When possible, begin making figures---even hand-drawn
sketches in your lab notebook. Remember to develop the overall
outline before writing specific sections. This helps to avoid
writing material that might later have to be discarded. As you write
portions of your thesis, show them to your advisor and to other
members in your research group for valuable feedback. Your advisor
will be much happier reviewing short pieces of your writing
periodically, as opposed to reviewing it all at once near the due
date. As you receive feedback along the way, you can apply it to
sections not yet written. The periodic feedback helps you to revise
and reshape your outline continually and guides you in developing a
clear scientific writing style. The writing process forces you to
organize your thoughts and to keep a clear vision of your research.
This helps you to avoid long periods of stagnation by bringing to
the foreground the next logical step in the research. The important
thing is to keep moving forward. No matter what, you will make many
mistakes, so try to make them as fast as you can. This is the
difference between experience and inexperience.

\section{Audience}
\label{sec:Audience}

You should consider as your audience the other students in your
research group. In particular, after you graduate, your thesis might
be used as a resource for students who will move into your former
role. Avoid making your thesis too basic; you may assume a certain
level of sophistication on the part of the reader. However, the
thesis should be easily understandable to a physics professor whose
research expertise is in a different field.

\section{Coherence}
\label{sec:Coherence}

Just as the overall outline of the thesis should have a clear and
logical organization, the sequence of information presented in each
section and paragraph should also follow a logical flow. Continually
ask yourself which paragraphs should appear before others. You
should be aware of a key sentence in each paragraph which usually
appears near the beginning and defines what the paragraph is
conveying. If a paragraph is very lengthy, don't hesitate to break
it into two (at a logical place). Read your own writing for logical
progression and for smoothness. Develop the skill of crafting smooth
transitions.

\section{Conciseness}
\label{sec:Concise}

Cut out the lard. Avoid long strings of prepositional phrases in
sentences of theses written by students in their senior year for the
physics department at BYU as a graduation requirement for the degree
of Bachelor of Science. (Did you get the joke?) Use simple
declarative sentences often, but not exclusively. Make every word
count. Vary sentence length. Intermingle short with long sentences
in an aperiodic fashion. You might inadvertently kill a reader with
boredom if your sentences all have the same length.

In your writing, be as quantitative as the subject matter permits,
and avoid inexact word usage. Continually ask yourself how your
writing might be misinterpreted. Make sure that arguments are
logically complete.

\section{Active voice}
\label{sec:ActiveVoice}

Remember that active verb construction generally captures the
reader's attention more than does passive construction. This does
not mean that passive voice should never be used. Just keep in mind
that an over reliance on passive verb construction results in a
rather bland document.

\section{Document format}
\label{sec:Format} \index{Format}

This document has been written following the format requested for
your thesis. Use a 12 point serif font such as Times for the main
text. If you desire variation, you can use a sans serif font such
as Helvetica for chapter and section headings. Set the margins to
one inch on all sides. Allow the right-hand side of the text to run
ragged if you use a regular word processor; text is easier to read
if it is not stretched and compressed in order to create a straight
right margin. LaTeX is a little fancier in its hyphenation and word
spacing and can usually pull of full justification fine. Double
space your lines. This makes the text easier to read and allows for
the insertion of mathematical expressions into the text without
disrupting the line spacing.

Use page breaks judiciously so that section headings do not become
isolated from their subsequent text on the bottom of a page.
Strategic positioning of figures within the text can help to avoid
large white spaces created when a figure's size forces it onto the
next page. If possible, you should avoid inserting figures before
they are discussed in the text. Your document should be printed
single-sided if it is less than 50 pages. However, if preferred you
may position figures on the backs of pages if this facilitates
keeping them close to the text describing them (located on the
adjacent page).

This document was prepared using LaTeX.  Many professors and
students choose to use LaTeX or variants such as REVTeX (a form used
by APS journals like Physical Review Letters). LaTeX can be
downloaded free of charge. For more information, go to the website
of the TeX Users Group (www.tug.org). Supplementary REVTeX macros
can be obtained from the website of the American Institute of
Physics (www.aip.org). You can also use Microsoft Word and MathType
(or the built in equation editor) for equations if you prefer. Check
with you advisor to find out which option will work best for you.

\section{Appropriate length}
\label{sec:Length}

How long does a senior thesis need to be? The answer is that it
should be as long as necessary to communicate your ideas succinctly.
There is no set length. The written document is not an end in
itself, but a vehicle to convey your ideas as efficiently as
possible to the reader. However, if the main body of your thesis
(excluding title pages and appendixes) is only 10 pages, you
probably have not included sufficient context and motivation for
your work. If the main body of your thesis exceeds 30 pages, you are
probably not concise enough. Professors and others don't really want
to read a lot of pages. A long thesis may also mean that you have
done more work than expected. Remember, you are not asked to
complete a masters project, rather a comparatively modest project in
your senior year. In the example in Table~\ref{table:outline}, the
introduction chapter might have 8 pages, the experimental setup
chapter might have 11 pages (emphasizing the work actually performed
by the hypothetical student in the example), and the results chapter
might have 6 pages (assuming results are obtained through a group
effort).

\section{Deadlines}
\label{sec:Deadlines} \index{Deadlines}

You must register for 2 credit hours of Physics 498R (499R if you
are in the honors program, 492R if you are in applied physics) in
any semester before graduation (usually when you are actively
involved in the research). At the end of the semester, if the thesis
is still in progress, you will receive a T grade. Then, when your
thesis is completed, your advisor will change it to a letter grade.
In order to allow for an adequate evaluation period, your thesis
should be submitted for review at least six weeks before graduation.
(Honor's theses are required ten weeks before graduation, submitted
first to the Honor's program and then indirectly to the department.)
After the thesis has been approved and signed by your advisor, take
it to the Senior Thesis Coordinator, currently Eric Hintz. If
acceptable, the Senior Thesis Coordinator will recommend your thesis
to the Department Chair for a signature. The Coordinator will then
forward the manuscript for binding and archiving in the Physics
Department library. Be sure to make an extra copy for yourself (and
one for your advisor). Typically, these additional private copies
are not bound unless you pay for it.
