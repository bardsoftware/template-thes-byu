\chapter{Getting Started}
\label{chap:GettingStarted}

\section{Choosing an advisor}
\label{sec:ChoosingAdvisor} \index{Choosing!an advisor}

You need a thesis advisor. Try to get one early, preferably before
your junior year. If you are unsure who to choose, you might want to
visit the department web page
(\href{http://www.physics.byu.edu/}{\url{http://www.physics.byu.edu}}) and click
on Research, which provides links to the various web sites for
faculty research. You can also click on Undergraduate and then
Research Opportunities to see the number of students working with
the different professors. Visit with prospective advisors during
their posted office hours or by appointment. Ask questions about
what research they are doing and how you might become involved.
Request a tour of any laboratory facilities that the professor uses.
Talk to other students who are currently doing research with the
professor. Keep in mind that establishing a connection with a
professor is a two-way process: You must choose an advisor, but
he/she must also choose you. Hint: Nothing pleases professors more
than if you attend their research group meetings (often held
weekly).

\section{Themes for projects}
\label{sec:ProjectThemes} \index{Choosing!a project}

When choosing an advisor (and after the choice is made), discuss
potential senior thesis projects at every opportunity. This will
prod your advisor to think about your specific case, and he/she will
more quickly recognize the right project for you in the research
group. Of course, you can give as much input as you like. However,
please recognize that professors in general are already committed to
certain research agendas. The best tack in the beginning is to
assist more experienced students with their research project as you
``learn the ropes." You will be amazed at how projects become more
interesting when you are involved with them. The problem that you
identify as the basis for \emph{your} senior thesis will soon become
extremely interesting.

\section{Financial support}
\label{sec:FinancialSupport} \index{Financial support}

Each summer the Department of Physics and Astronomy supports over a
dozen students at 20 hours per week while they work on their senior-
thesis research. This means that you can get paid for fulfilling
your graduation requirement. To be selected for this, you must
submit a simple online proposal,usually due in late February. To
apply, go to the department website click on Undergraduate and then
Student Employment.  In addition, the department supports a limited
number of students at about 10 hours per week to do research during
the Fall and Winter Semesters. Many professors have research funds
that are earmarked to pay undergraduate research assistants.

The Office of Research and Creative Activities (ORCA) also holds a
scholarship competition each fall for students involved in research.
The award amount is about \$1,500 and you can get this on top of the
summer or school-year funding. The application deadline is usually
early October, and you may apply online. From the BYU home page,
choose Students, then Academic Lenis, and then Research Support. For
those doing an Honors Thesis, there are some funds available through
the Honors Program. Qualifying minority students can apply (through
their professor) for funding from the Western Alliance to Expand
Student Opportunities (WAESO): (\href{http://www.asu.edu/WAESO}
{\url{http://www.asu.edu/WAESO}}).

Finally, you should be aware of the \emph{Research Experiences for
Undergraduates} (REU) program sponsored by the National Science
Foundation. Many universities across the nation participate in this
program (including BYU). These universities host undergraduates
(mainly from other institutions) and involve them in research during
a few summer months. Applications are usually due in February. Visit
the National Science Foundation web site of \emph{Student Interests}
(\href{http://www.nsf.gov/crssprgm/reu/}{\url{http://www.nsf.gov/crssprgm/reu/}}).
Under the heading \emph{For Undergraduate Students}, click on
\emph{list of REU sites} for a complete listing of sites and contact
information.  Please be aware that there are also international REU
opportunites as well, which a Google search can help identify.

It is possible to use your REU research away from BYU as the basis
for your senior thesis. If you do, try as much as possible to
prepare your thesis \emph{while} you are participating in the REU
program. Ask the professor with whom you are working to help you to
revise your thesis during the visit. Get started early to allow time
for revisions. As part of the REU program, you will be required to
write a report on your research experience. While a thesis is much
more than a report, you can use your thesis (or portions thereof) in
the report. For purposes of completing your senior thesis, you will
need a thesis advisor at BYU. Choose one before you go to the REU
site. After you return, your BYU advisor can help you to revise your
thesis and see that it is satisfactory for submission to the
department.

\section{Literature}
\label{sec:Literature}
\subsection{Using the library}
\label{ssec:UsingLibrary} \index{Library resources}

The library holds enormous resources, many of which are probably
unfamiliar to you. Contact John Christensen (2323 HBLL, 422-2928)
who is the Library Subject Specialist for Physics and Astronomy. He
holds training sessions periodically on how to use library resources
\emph{for physics research}. \textbf{Attend a training session}. He
gets paid to teach you. Get as many of your peers in the department
as you can to go with you. This training can be more valuable
than five quantum-mechanics lectures!

\subsection{Academic journals}
\label{ssec:AcademicJournals} \index{Literature!journals}

New physics research is mainly published in journals, rather than in
textbooks. It takes time for the more relevant information to find
its way into textbooks. As you study the relevant issues in your
field of research, it is essential that you study articles in
academic journals. Some of the more prestigious physics journals are
Physical Review A, B, C, D, and E, and Physical Review Letters,
which are published by the American Physical Society (APS). There
are dozens of other reputable physics journals, often emphasizing
specialized areas, almost all of which can be found in the large new
atrium underground wing of the HBLL. However, it is rarely necessary
to visit the library to find articles, since almost all of them
journal articles can be accessed online. The library pays
subscription fees that enables electronic access to the journal
archives from any BYU-campus computer.

Journals are collections of scientific articles that undergo
scrutiny through an anonymous peer review system. Physicists mainly
publish articles about their research in these types of journals.
Therefore, journals should be the primary source of background and
contextual information in your thesis, as opposed to web sites, for
example, which are probably not peer reviewed. Conference
proceedings are another important source of information. They are
collections of short articles submitted by participants at
scientific conferences.

\subsection{Using search engines}
\label{ssec:SearchEngines} \index{Search engines}

An electronic search engine is by far the most important tool for
finding relevant research information. One of the easiest to use is
\href{http://scholar.google.com/}{\url{http://scholar.google.com}}. Just
type in key words or author names and see what comes up. Often there
are links to the journal archives, which usually automatically know
that you are connected through BYU (a subscriber), and so they allow
you to download articles. Other important search engines can be
accessed through the Harold B. Lee Library web site
(\href{http://www.lib.byu.edu/}{\url{http://www.lib.byu.edu}}).
Choose Find Articles and then select a subject such as Physics. Two
excellent search engines are SPIN and Web of Science.

You should be aware that BYU pays a lot of money for access to
search engines such as SPIN and Web of Science. BYU pays this money
partly for \emph{your} sake, so please take advantage of the
service. There are other search databases which may wish to try. It
is important to use several to make sure to get good coverage in
your search.  For example, although SPIN is very nice to use, it
does not return results from European journals.  There are some very
good physics search engines such as INSPEC to which BYU does not
subscribe (because it is very expensive). Contact John Christensen
(2323 HBLL, 422-2928) to be directed to a librarian who can run
searches on INSPEC for you. Short searches are complimentary to
students, but there is a modest fee for lengthy searches.

\section{Effective searching}
\label{sec:EffSearch} \index{Search engines!using effectively}

When beginning a search, you may initially want to restrict it to
articles published in recent years (say, last 10 years) in order to
avoid getting deluged with ``hits." (The library search engines
give you much more control of your search parameters than does
scholar.google.com.) You will probably want to begin
the search using key subject words. As you locate relevant titles,
read the abstracts to decide which articles are important to you. As
you find relevant articles, follow up with searches using author
names. This may turn up additional related articles.  As you search,
you will likely encounter dozens of potentially relevant articles. A good
search will probably take a few hours as you read over many abstracts.

You will be able to download most articles to your computer in PDF format.
If a journal is unavailable, you can order a copy of an article
through the Interlibrary Loan Service (free to you, but not free to
the library).  After acquiring a few articles,
begin to read them (or skim) to further assess relevance. Especially
pay attention to the introduction where authors summarize the
relevant publications of others. Doing this, you will find many new
references. This is an efficient and very important way to network
your search back in time. You are relying on other experts in the
field to point you towards seminal research articles. As you find
these important papers, use the Science Citation Index, which will
point you to papers that have referenced them. This is the way to
network forward in time.

Take your time as you search for articles in your field. A poor
strategy is to download blindly a long list of potentially useful
articles. Rather, decide whether articles are relevant as you go.
Make notes to yourself about the content of different papers.
Include any references that you received from your advisor in your
search from the very beginning. Those articles will often be the
most important. After many hours of using search engines and finding
articles in the library, you may have considered more than 50
different articles and found more than a dozen that are very
relevant to your research. You will likely summarize a number of
these in your thesis introduction and refer to many of them at
relevant points throughout your thesis. You may have occasion to
refer to a few books as well, so don't forget to do library searches
for books related to your topic.

\section{Reading and understanding the literature}
\label{sec:ReadingLit} \index{Literature!understanding}

Don't be discouraged when attempting to read physics research
articles. Feeling utterly lost is quite normal, even for experienced
physicists. It takes time to penetrate physics articles, since the
information is often presented very compactly, intended for other
experts in the field. It is helpful to initially read only the
abstract, the introduction, and the conclusion. You might save the
interior of important papers for a discussion with your advisor,
perhaps during a group meeting. As you read more scientific papers,
you will acquire a feel for the overall structure and flow, the
manner of documentation and reasoning. In fact, it is exactly this
efficient and well-mannered approach to technical writing that
should go into your thesis (with a somewhat different audience in
mind).

If you struggle to understand articles, you may have trouble
with more than just the physics. Ask someone to read and discuss the article with you.
You may be surprised at how well a good reader, even
one without a physics background, can identify key issues and conclusions.
